\documentclass[a4peper, 12pt, titlepage, finall]{extreport}

%различные пакеты

\usepackage[T1, T2A]{fontenc}
\usepackage[russian]{babel}
\usepackage[backend=bibtex]{biblatex}
\usepackage{csquotes}
\usepackage{tikz}
\usepackage{geometry}
\usepackage{indentfirst}
\usepackage{fontspec}
\usepackage{graphicx}
\usepackage{array}
\graphicspath{{./images/}}

\usetikzlibrary{positioning, arrows}

\geometry{a4paper, left = 15mm, top = 10mm, bottom = 15mm, right = 15mm}
\setmainfont{Spectral Light}%{Times New Roman}
%\setmonofont{Courier New}
\setcounter{secnumdepth}{0}
%\setcounter{tocdepth}{3}
\nocite{*}
\begin{document}
\begin{center}
    {\large \bf «Исследование применимости алгоритмов сжатия данных к таблицам классификации в сетевом процессоре.»}

\end{center}
        \begin{flushright}
            {Никифоров Никита Игоревич, 421 группа}\\
            {Научные руководители:\\ Волканов Д. Ю., Скобцова Ю. А.}
        \end{flushright}
    \section{Отчёт за неделю 16.10 - 23.10}
        Было выделено два основных подхода к решению поставленной задачи на основе прочитанных статей.
        Первых подход заключается в обычном сжатии таблиц классификации на уровне ЦПУ в коммутаторе и их дальнейшая распаковка
        и использование. Данный подход не принесёт ожидаемых результатов уменьшения затраченной памяти на СП.
        Таким образом необходимо рассматривать только алгоритмы, которые позволят сразу строить сжатое дерево в СП,
        по которому будет происходить процесс классификации.

        Алгоритм сжатия описанный на данный момент в формальной постановке задачи является первичным и благодаря ему уменьшается
        количество взаимозаменяемых правил. Искомый же алгоритм сжатия должен строить оптимальную таблицу классификации.
        Таким образом выходная структура данных о которой шёл разговор на прошлой встречи не имеет смысла, так как алгоритм построения
        дерева и алгоритм сжатия будут взаимо дополнять друг друга.

        Таким образом я пришёл к понятию оптимальности алгоритма сжатия. Соответственно необходимо поставить формальную постановку 
        задачи оптимизации. (Хороший вариат я видел на статье на диске)

        \subsection{Проделанная работа}
            За прошёдшую неделю мною была проделана следующая работа:
            \begin{itemize}
                \item Прочитаны три статьи.
                \item Проведён анализ различных подходов решения поставленной задачи.
                \item Разработаны первичные критерии выбора алгоритмов сжатия.
                \item Подправлен план на текущий курс.
            \end{itemize}
        \subsiction{Первичные критерии обзора}
            В рамках данной работы будут рассматриваться только алгоритмы без потерь данных при сжатии.
            Также в алгоритмах необходимо учитывать частоту используемых правил.
            \begin{itemize}
                \item Степень сжатия
                \item Ассимтотическая сложность сжатия
                \item Однозначность декодировки
            \end{itemize}
\end{document}
