\begin{lmrarticle}
{Исследование применимости алгоритмов сжатия к таблицам потоков в сетевом процессоре RuNPU}
    {Никифоров~Н.\,И., доцент\,к.ф.-м.н.~Волканов~Д.\,Ю.}
\TwoAuthor%
{Никифоров Никита Игоревич}
    {Кафедра автоматизации систем вычислительных комплексов}{nickiforov.nik@gmail.com}
{Волканов Дмитрий Юрьевич}
    {Кафедра автоматизации систем вычислительных комплексов}{volkanov@asvk.cs.msu.ru}

В настоящее время активно развиваются технологии программно-конфигурируемых сетей (ПКС)~\cite{smelopen}. Для работы ПКС требуются высокопроизводительные коммутаторы, 
которые выполняют функцию передачи данных. Возникает задача разработки программируемого сетевого процессорного устройства (СПУ),
являющегося основным функциональным элементом коммутаторов.

В работе рассматривается коммутатор функционирующий под управлением протокола\\ OpenFlow.
Правила обработки пакетов в котором представляются в виде таблицы потоков, групповые таблицы рассматриваться не будут.
В СПУ таблицы потоков представляются в виде программы обработки заголовков сетевых пакетов.
Для преобразования таблиц потоков в программу обработки заголовков сетевых пакетов используется транслятор табиц потоков.

СПУ представляет из себя интегральную микросхему. В рассматриваемом СПУ (RuNPU) применяется конвейерная архитектура,
а именно на каждый входной порт коммутатора $-$ СПУ содержит конвейер, состоящий из вычислительных блоков. Каждый вычислительный блок имеет доступ к 
устройству памяти в котором хранится программа обработки заголовков сетевых пакетов. Рассматриваемый СПУ имеет ограниченный объём доступной
памяти, для хранения программы обработки заголовков сетевых пакетов.
Современные таблицы потоков занимают до нескольких десятков мегабайтов памяти \cite{rottenstreich2016optimal}. Поэтому возникает задача сжатия таблиц потоков,
для использования рассматриваемого СПУ в коммутаторах ПКС.

В рамках работы проведён обзор существующих алгоритмов сжатия, в котором были учтены следующие ограничения рассматриваемого СПУ:
необходимость использования сжатых таблиц потоков без декомпрессии, ограниченный объём памяти на конвейерах СПУ. Критериями в обзоре
были выбраны: возможность использования сжатых таблиц потоков без декомпрессии, оценка степени сжатия, необходимость использования внешней памяти
для работы алгоритмов сжатия. На основе обзора был выбран алгоритм оптимального кеширования, с использованием внешней памяти.

Для проведения экспериментального исследования выбранный алгоритм был реализован в трансляторе таблиц потоков, также была добавлена поддержка внешней памяти в 
эмулятор СПУ. Использование алгоритма оптимального кеширования позволило снизить объём затрачиваемой памяти СПУ.\\
\begin{lmrreferences}
\end{lmrreferences}
\end{lmrarticle}
