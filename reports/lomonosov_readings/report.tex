\begin{lmrarticle}
{Исследование применимости алгоритмов сжатия к таблицам потоков в сетевом процессоре RuNPU}
    {Никифоров~Н.\,И., доцент\,к.ф.-м.н.~Волканов~Д.\,Ю.}
\TwoAuthor%
{Никифоров Никита Игоревич}
    {Кафедра автоматизации систем вычислительных комплексов}{nickiforov.nik@gmail.com}
{Волканов Дмитрий Юрьевич}
    {Кафедра автоматизации систем вычислительных комплексов}{volkanov@asvk.cs.msu.ru}

    В настоящее время активно развиваются технологии\\ программно-конфигурируемых сетей (ПКС)~[1]. Для работы ПКС требуются высокопроизводительные коммутаторы, 
которые выполняют функцию передачи данных. Возникает задача разработки программируемого сетевого процессорного устройства (СПУ),
являющегося основным функциональным элементом коммутаторов.

В СПУ таблицы потоков OpenFlow представляются в виде программы обработки заголовков сетевых пакетов.
Для преобразования таблиц потоков в программу обработки заголовков сетевых пакетов используется транслятор табиц потоков.

В рассматриваемом СПУ (RuNPU) применяется конвейерная архитектура. Конвейер состоит из вычислительных блоков. Каждый вычислительный блок имеет доступ к 
устройству памяти в котором хранится программа обработки заголовков сетевых пакетов. Рассматриваемый СПУ имеет ограниченный объём доступной памяти. Современные таблицы потоков занимают до нескольких десятков мегабайтов памяти~[2, 3]. Поэтому возникает задача сжатия таблиц потоков,
для использования рассматриваемого СПУ в коммутаторах ПКС.

В рамках работы проведён обзор существующих алгоритмов сжатия, в котором были учтены следующие ограничения рассматриваемого СПУ:
необходимость использования сжатых таблиц потоков без декомпрессии, ограниченный объём памяти на конвейерах СПУ. На основе обзора был выбран алгоритм оптимального кеширования, с использованием внешней памяти.

Для проведения экспериментального исследования выбранный алгоритм был реализован в трансляторе таблиц потоков, также была добавлена поддержка внешней памяти в 
эмулятор СПУ. Использование алгоритма оптимального кеширования позволило снизить объём затрачиваемой памяти СПУ.\\
\begin{lmrreferences}
\item
Семелянский\,Р.\,Л. Програмно-конфигурируемые сети.
Журнал <<Открытые системы>>
(2012\,г.).Том:~9 C.\, 15--26.
\item
Rottenstreich,~Ori~and~Tapolcai,~J{\'a}nos.
Optimal rule caching and lossy compression for longest prefix matching.
IEEE/ACM Transactions on Networking.
(2016\,г.).~C. 864--878.~IEEE.
\item
Chang\,~Yeim-Kuan~and~Chen\,~Han-Chen.
Fast packet classification using recursive endpoint-cutting and bucket compression on FPGA.
The Computer Journal.
    (2019\,г.).~C. 198--214.~Oxford University Press.
\end{lmrreferences}
\end{lmrarticle}
