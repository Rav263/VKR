\documentclass[a4peper, 12pt, titlepage, finall]{extreport}

%различные пакеты

\usepackage[T1, T2A]{fontenc}
\usepackage[russian]{babel}
\usepackage[backend=bibtex]{biblatex}
\usepackage{csquotes}
\usepackage{tikz}
\usepackage{geometry}
\usepackage{indentfirst}
\usepackage{fontspec}
\usepackage{graphicx}
\usepackage{array}
\graphicspath{{./images/}}

\usetikzlibrary{positioning, arrows}

\geometry{a4paper, left = 15mm, top = 10mm, bottom = 15mm, right = 15mm}
\bibliography{problem}
\setmainfont{Spectral Light}%{Times New Roman}
%\setmonofont{Courier New}
\setcounter{secnumdepth}{0}
%\setcounter{tocdepth}{3}
\nocite{*}
\begin{document}
\begin{center}
    {\large \bf «Исследование применимости алгоритмов сжатия данных к таблицам классификации в сетевом процессорном устройстве.»}

\end{center}
        \begin{flushright}
            {Никифоров Никита Игоревич, 421 группа}\\
            {Научные руководители:\\ Волканов Д. Ю., Скобцова Ю. А.}
        \end{flushright}
    \section{Цель и задачи обзора}
        Целью данного обзора является выбор алгоритмов сжатия для применения в сетевом процессорном устройстве. 
        Необходимость применения алгоритмов сжатия обусловлена недостатком объёма памяти конвейера сетевого процессорного устройства.
        
        В настоящем обзоре будут использоваться следующие критерии:
        \begin{enumerate}
            \item Отношение объёма памяти занимаемого таблицей классификации после применения алгоритма сжатия, 
                к изначальному объёму памяти занимаемому таблицей классификации (Степень сжатия).
            \item Оценка сложности сжатия.
            \item Возможность использования без декомпрессии таблиц классификации $-$ 
                из-за ограничений рассматриваемого сетевого процессорного устройства сжатые таблицы классификации 
                должны представляться как код ассемблера.
            \item Использование внешней памяти $-$ необходимость использования внешней памяти 
                для использования сжатых таблиц классификации без декомпрессии.
        \end{enumerate}

    \section{Рассматриваемые алгоритмы сжатия}
        \subsection{Распространённые алгоритмы}
            Под распространёнными алгоритмами сжатия будем понимать алгоритмы, 
            которые сжатые данные представляют в бинарном виде. 
            Примером таких алгоритмов может служить:
            \begin{itemize}
                \item алгоритм Хаффмана,
                \item JPEG,
                \item LWZ,
                \item zip.
            \end{itemize}
            
            Рассмотрим критерии для данного класса алгоритмов. Различные алгоритмы в данной группе имеют различную степень сжатия, 
            которая колеблется от 0.3 до 0.8. Большинство рассматриваемых алгоритмов имеют квадратичную или кубическую сложность сжатия 
            от объёма сжимаемых данных. Так как сжатые данные после применения алгоритмов из данной группы представляются в бинарном виде,
            невозможно использовать сжатые таблицы классификации без декомпрессии.
        \subsection{Алгоритм оптимального кеширования}
            Данный алгоритм основан на построение дерева правил относительно их частот, остальные правила хранятся на центральном процессоре коммутатора.
            Количество правил, которые хранятся в сетевом процессорном устройстве, вычисляется по формуле:

            Рассмотрим критерии для данного алгоритма: Степень сжатия данного алгоритма зависит от частот использования префиксов, от 0.1 до 0.9.
            Данный алгоритм имеет квадратичную сложность построения. Сжатые таблицы классификации возможно использовать без декомпрессии, 
            но появляются накладные расходы при обращении к внешней памяти. 
        \subsection{Алгоритм Recursive endpoint cutting}
            
    \begingroup
    \let\clearpage\relax
    \printbibliography
    \endgroup
\end{document}
