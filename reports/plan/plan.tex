\documentclass[a4peper, 12pt, titlepage, finall]{extreport}

%различные пакеты

\usepackage[T1, T2A]{fontenc}
\usepackage[russian]{babel}
\usepackage[backend=bibtex]{biblatex}
\usepackage{csquotes}
\usepackage{tikz}
\usepackage{geometry}
\usepackage{indentfirst}
\usepackage{fontspec}
\usepackage{graphicx}
\usepackage{array}
\graphicspath{{./images/}}

\usetikzlibrary{positioning, arrows}

\geometry{a4paper, left = 15mm, top = 10mm, bottom = 15mm, right = 15mm}
\setmainfont{Spectral Light}%{Times New Roman}
%\setmonofont{Courier New}
\setcounter{secnumdepth}{0}
%\setcounter{tocdepth}{3}
\nocite{*}
\begin{document}
\begin{center}
    {\large \bf «Исследование применимости алгоритмов сжатия данных к таблицам классификации в сетевом процессоре.»}

\end{center}
        \begin{flushright}
            {Никифоров Никита Игоревич, 421 группа}\\
            {Научные руководители:\\ Волканов Д. Ю., Скобцова Ю. А.}
        \end{flushright}
    \section{План работы на 4-ый курс.}
        \subsection{План работы на осенний семестр}
            \begin{enumerate}
                \item 09.10.20 - 16.10.20
                    \subitem Написать план работы на курс
                    \subitem Написать формальную постановку задачи
                    \subitem Продолжить работу со статьями
                \item 16.10.20 - 23.10.20
                    \subitem Внесение правок в формальную постановку задачи
                    \subitem Формирование критериев обзора алгоритмов
                    \subitem Начало написания обзора алгоритмов
                    \subitem Продолжить работу со статьями
                \item 23.10.20 - 30.10.20
                    \subitem Продолжение обзора алгоритмов сжатия
                    \subitem Формализация оптимизационной постановки задачи
                    \subitem Разработка основной концепции изменений сущестующей модели СП
                    \subitem Продолжить работу со статьями
                    \subitem Выступление на конференуии monetec
                    \subitem Работа со статьёй в сборник кафедры
                \item 30.10.20 - 06.11.20
                    \subitem Начало написания текста обзора алгоритмов сжатия
                    \subitem Формализация выбранного алгоритма сжатия
                    \subitem Подготовка презентации и выступления на спец. семинаре
                    \subitem Продолжение работы со статьёй в сборник кафедры
                \item 06.11.20 - 13.11.20
                    \subitem Формализация изменений в существующей модели сетевого процессора
                    \subitem Разработка структуры модуля алгоритмов сжатия в модели сетевого процессора
                    \subitem Внесение правок в текст обзора алгоритмов сжатия
                    \subitem Дальнейшая подготовка к выступлению на спец семинаре и выступление на нём
                    \subitem Продолжение работы со статьёй в сборник кафедры
                \item 13.11.20 - 20.11.20
                    \subitem Начало разработки выбранных в рамках обзора алгоритмов сжатия
                    \subitem Разработка методики экспериментального исследования
                    \subitem Подготовка к выступлению на конференции Ломоносов 
                \item 20.11.20 - 27.11.20
                    \subitem Продолжение разработки методики экспериментального исследования
                    \subitem Написание текста зимного отчёта
                \item 27.11.20 - 04.12.20
                    \subitem Подготовка выступления на зимнем отчёте по ВКР
                    \subitem Внесение правок в методику экспериментального исследования
            \end{enumerate}
        \subsection{План работы на весенний семестр}
            \begin{itemize}
                \item Конец января - начало февраля
                    \subitem Отдых от сессии
                    \subitem Разработка изменений в эмулятор сетевого процессора
                    \subitem Работа с текстом ВКР
                    \subitem Разработка алготмов сжатия
                \item Конец февраля - март
                    \subitem Завершение разработки алгоритмов сжатия
                    \subitem Сбор данных для проведения экспериментального исследования
                    \subitem Окончание разработки изменений в модель сетевого процессора
                    \subitem Окончание работы с текстом (кроме экспериментального исследования)
                \item Апрель
                    \subitem Проведение экспериментального исследования
                    \subitem Звершение написания текста ВКР
                    \subitem Подготовка к выступлению на защите ВКР
            \end{itemize}
    \section{Работы в 8-м семестре}
        \subsection{10.02.21 - 17.02.21}
            {\bf Что планировалось:}
            \begin{itemize}
                \item Получить версию транслятора
                \item Разбираться с кодом транслятора
                \item Продумывать архитектуру
            \end{itemize}
            {\bf Что сделано:}
            \begin{itemize}
                \item Получен транслятор от Андрея
                \item Начал разбираться с транслятором
                \item Появились некоторые вопросы по архитектуре транслятора.
            \end{itemize}
        \subsection{17.02.21 - 26.02.21}
            {\bf Что планируется:}
            \begin{itemize}
                \item Разобраться с архитектурой транслятора.
                \item Подготовить архитектуру для внедрения алгоритмов сжатия.
                \item Попробовать внедрить заглушку.
                \item Работа с текстом ВКР (20.02.21).
            \end{itemize}
            {\bf Что сделано:}
            \begin{itemize}
                \item Разобрался с архитектурой трансялтора.
                \item Приступил к реализации алгоритма оптимального кеширования.
                \item Обновил текст ВКР (Разделы обзора, системы трансляции, экспериментального исследования).
            \end{itemize}
        \subsection{26.02.21 - 03.03.21}
            {\bf Что планируется:}
            \begin{itemize}
                \item Завершить реализацию алгоритма оптимального кеширования в системе трансляции.
                \item Начать реализацию алгоритма REC.
                \item Работа с текстом ВКР
            \end{itemize}
           {\bf Что сделано:}
            \begin{itemize}
                \item Обновил текст ВКР.
            \end{itemize}
        \subsection{05.03.21 - 12.03.21}
            {\bf Что планируется:}
            \begin{itemize}
                \item Завершить реализацию алгоритма оптимального кеширования в системе трансляции.
                \item Начать реализацию алгоритма REC.
                \item Работа с текстом ВКР
            \end{itemize}
            {\bf Что сделано:}
            \begin{itemize}
                \item Был обновлён текст ВКР.
            \end{itemize}
        \subsection{23.04.21 - 30.04.21}
            {\bf Что планируется:}
            \begin{itemize}
                \item Завершить презентацию к предзащите
                \item Обновить текст ВКР
                \item Продолжать экспериментальное исследование
                \item Предзащитить ВКР
            \end{itemize}
            {\bf Что сделано:}
            \begin{itemize}
                \item Предзащита пройдена успешно
                \item Обновлён текст ВКР, в соответствии с замечаниями на предзащите
                \item Продолжено экспериментальное исследование
            \end{itemize}
        \subsection{30.04.21 - 07.05.21}
            {\bf Что планируется:}
            \begin{itemize}
                \item Продолжать писать текст ВКР
                \item Продолжать экспериментальное исследование
                \item Обновить текст статьи
            \end{itemize}
            {\bf Что сделано:}
            \begin{itemize}
                \item Обновлён текст ВКР
                \item Обнаружена неточность в реализации в эмуляторе
                \item Обновлён текст статьи
            \end{itemize}
\end{document}
