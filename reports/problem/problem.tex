\documentclass[a4peper, 12pt, titlepage, finall]{extreport}

%различные пакеты

\usepackage[T1, T2A]{fontenc}
\usepackage[russian]{babel}
\usepackage[backend=bibtex]{biblatex}
\usepackage{csquotes}
\usepackage{tikz}
\usepackage{geometry}
\usepackage{indentfirst}
\usepackage{fontspec}
\usepackage{graphicx}
\usepackage{array}
\graphicspath{{./images/}}

\usetikzlibrary{positioning, arrows}

\geometry{a4paper, left = 15mm, top = 10mm, bottom = 15mm, right = 15mm}
\bibliography{problem}
\setmainfont{Spectral Light}%{Times New Roman}
%\setmonofont{Courier New}
\setcounter{secnumdepth}{0}
%\setcounter{tocdepth}{3}
\nocite{*}
\begin{document}
\begin{center}
    {\large \bf «Исследование применимости алгоритмов сжатия данных к таблицам классификации в сетевом процессоре.»}

\end{center}
        \begin{flushright}
            {Никифоров Никита Игоревич, 421 группа}\\
            {Научные руководители:\\ Волканов Д. Ю., Скобцова Ю. А.}
        \end{flushright}
    \section{Актуальность}
        В данной работе рассматривается архитектура сетевого процессора (СП), в которой используется конвейерная архитектура.
        Конвейер состоит из последовательных вычислительных блоков, в каждом из которых находится независимое устройство памяти.
        В памяти вычислительного блока хранится программа классификации пакетов. Современные таблицы потоков занимают
        до нескольких десятков мегабайтов памяти \cite{rottenstreich2016optimal}.
        В связи с малым объёмом памяти внутри одного вычислительного блока $-$ 64 Кб, 
        необходимо провести исследование существующих алгоритмов сжатия данных, и предложить их адаптацию для использования 
        в рассматриваемом СП.
    \section{Не формальная постановка задачи}
        Необходимо исследовать применимость существующих алгоритмов сжатия данных в существующей архитектуре СП.
        Рассматриваемые алгоритмы должны удовлетворять следующим условиям:
        \begin{itemize}
            \item Размер итоговой таблицы потоков не должен превышать 512 Кб.
            \item Потери данных при использования алгоритмов сжатия не должны быть значительными.
        \end{itemize}
    \section{Формальная постановка задачи}

        Введём формализацию OpenFlow таблиц.
        Упорядоченное множество всех рассматриваемых признаков в правилах обозначим \(I=\{m_1,m_2,\ldots,m_k\}\). 
        Каждый признак \(m_i\) из множества признаков \(I\) характеризуется конечным множеством значений \(D(m_i)\subset\mathbb{Z_+}\). 
        Признак также может принимать значение <<\textit{любой}>> (далее~--- \(*\)).

        Представим таблицу потоков в виде множества правил \(R=\{r_1,r_2,\ldots,r_n\}\). С каждым правилом \(r_i\) связаны:
        \begin{itemize}
            \item номер \(i\);
            \item приоритет \(p_i\in \mathbb{Z_+}\);
            \item вектор значений признаков \(f_i=\{f_i^1,f_i^2,\ldots,f_i^k\}\), где \(f_i^j\) соответствует значению признака \(m_j\in I\) и \(f_i^j\in D(m_j)\cup\{*\}\), \(j=\overline{1,k}\).
        \end{itemize}

        Будем говорить, что заголовок пакета и его метаданные с вектором значений признаков \(g=\{g^1,g^2,\ldots,g^k\}\) 
        соответствуют правилу \(r_i\in R\) с вектором значений признаков \(f_i=\{f_i^1,f_i^2,\ldots,f_i^k\}\) 
        и приоритетом \(p_i\) (правило \(r_i\in R\) идентифицирует пакет с вектором знаений признаков \(g\)), если:

        \begin{enumerate}
            \item вектор значений признаков \(g\) соответствует вектору значений признаков \(f_i\), то есть \(f_i^l\in\{*,g^l\}\), \(l=\overline{1,k}\);
            \item приоритет \(p_i\) максимален среди всех правил \(r_j\in R\), для которых \(g\) соответствует вектору значений признаков \(f_j\).
        \end{enumerate}

        Множество \(R\) также должно удовлетворять следующему ограничению. 
        Для любых двух правил \(r_i,r_j\in R,r_i\not= r_j\), если их вектора значений пересекаются, то есть существует набор значений признаков, 
        который соответствует векторам значений признаков обоих правил, то \(p_i\not= p_j\). 
        Например, правила с векторами значений признаков \(f_i=\{1, 2, *\}\) и \(f_j=\{*, 2, 3\}\) должны иметь разный приоритет, 
        так как набор значений признаков \(g=\{1, 2, 3\}\) соответствует обоим правилам.

        Введём понятие анологичности множеств \(R_1\) и \(R_2\).
        Множество \(R_1\) аналогично мноеству \(R_2\), если для любого пакета, для которого существует идентифицурующее его правило в множестве \(R_1\), 
        найдётся правило идентифицирующее его в множестве \(R_2\).

        Необходимо разработать алгоритм сжатия таблиц потоков, который будет переводить исходное множество $-$ \(R_1\), соответствующее исходной таблице потоков, в
        новое множество \(R_2\), которое соответствует новой таблице потоков.
        \begin{enumerate}
            \item Множество \(R_1\) должно быть анологично множеству \(R_2\).
            \item Мощность множества \(R_2\) должна быть меньше либо равно мощности множества \(R_1\).
        \end{enumerate}
    \begingroup
    \let\clearpage\relax
    \printbibliography
    \endgroup
\end{document}
