\documentclass[conference]{IEEEtran}

%\usepackage[russian]{babel}
\usepackage[justification=centering]{caption}
\usepackage[backend=bibtex]{biblatex}
\usepackage{fontspec}
\usepackage[final]{graphicx}
\usepackage{float}
\usepackage{subcaption}
\usepackage{listings}
\usepackage{array}
\usepackage{xcolor}
\usepackage{graphicx}
\usepackage{tikz}
\usepackage{pgfplots}
%\usepackage{geometry}
\graphicspath{{./images/}}

\setmainfont{Spectral Light}%{Times New Roman}

\definecolor{codegreen}{rgb}{0,0.6,0}
\definecolor{codegray}{rgb}{0.5,0.5,0.5}
\definecolor{codepurple}{rgb}{0.58,0,0.82}
\definecolor{backcolour}{rgb}{0.95,0.95,0.92}

\lstdefinestyle{mystyle}{
    backgroundcolor=\color{white},   
    commentstyle=\color{codegreen},
    keywordstyle=\color{magenta},
    numberstyle=\color{codegray},
    stringstyle=\color{codepurple},
    basicstyle=\ttfamily\footnotesize,
    morekeywords={*,procedure, if, rol, cmpj, setmask, then, else, endif, cmpjn, is, not, and, return},            % if you want to add more keywords to the set
    breakatwhitespace=false,         
    breaklines=true,                 
    captionpos=b,                    
    keepspaces=true,                 
    numbers=left,                    
    numbersep=10pt,
    xleftmargin=7mm,
    xrightmargin=0mm,
    showspaces=false,                
    showstringspaces=false,
    showtabs=false,                  
    tabsize=4
}
\lstset{style=mystyle}
\lstset{linewidth=9cm}
\bibliography{syrcose} 


\begin{document}
\author{
    \IEEEauthorblockN{Nikita Nikiforov}
    \IEEEauthorblockA{\textit{Lomonosov Moscow State University}\\
    Moscow, Russia\\
    nickiforov.nik@gmail.com}
    \and
    \IEEEauthorblockN{Dmitry Volkanov}
    \IEEEauthorblockA{\textit{Lomonosov Moscow State University}\\
    Moscow, Russia\\
    volkanov@asvk.cs.msu.ru}
    }
\title{
    Data compression algorithms for flow tables in Network Processor (RuNPU).
}
\maketitle
    \begin{abstract}
        This paper addresses the problem of packet classification within a network processor (NP) architecture 
        without the separate associative device.
        By the classification, we mean the process of identifying a packet by the header.
        The classification stage requires the implementation of data structures to store the flow tables.
        In our work we consider NP without the associative memory. In considering NP flow tables
        represent like an assembly language program. For translating flow tables into assembly language programs,
        tables translator was used. 
        Nowadays flow tables can take tens of megabytes of memory.
        This is the reason for implement data compression algorithms in flow table translator.
        In this work we provide the data compression algorithms: Optimal rule caching, recursive end-point cutting and
        bit string compression. An evaluation of the implemented data compression algorithms was performed on a 
        simulation model of the NP.
    \end{abstract}
    
    \begin{IEEEkeywords}
        Network processor, software-defined networks, packet classification, data compression.
    \end{IEEEkeywords}

    \section{Introduction}
        At present, software-defined networks (SDN) are in active developing and require high-performance switches.
        The main functional element of the high-performance SDN switch is programmable network processor.
        The network processor is system-on-chip specialized for network packet processing. In our work we consider
        the programmable NP. By programmable we mean such NP, that supports to change the packet processing program 
        and the set of processed header fields on the fly. 
        
        This article will discuss about data compression algorithms used for flow tables. 
        Flow tables needs for packet classification process. Flow table is the set of flows, that defines
        by OpenFlow protocol. Each rule contains match field, bit string by witch a packet can be identified 
        and set of actions, that NP performs on this packet.
        The classification is the process of identification a network packet by it's header. 
        
    \section{Network processor architecture}
    \section{The problem}
    \section{Related work}
    \section{Our solution}
    \section{Evaluation}
    \section{Future work}
 \printbibliography{}
\end{document}
